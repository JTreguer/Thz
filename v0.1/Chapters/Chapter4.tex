% Chapter Template

\chapter{Corrélations en cosmologie} % Main chapter title

\label{Chapter4} % Change X to a consecutive number; for referencing this chapter elsewhere, use \ref{ChapterX}

\lhead{Chapter X. \emph{Chapter Title Here}} % Change X to a consecutive number; this is for the header on each page - perhaps a shortened title

%----------------------------------------------------------------------------------------
%	SECTION 1
%----------------------------------------------------------------------------------------

\section{Définition}
\subsection{Auto-corrélation}
\subsection{Corrélation croisée}
\section{Les mesures de corrélations en cosmologie}
\subsection{Fonction à deux points}
\subsection{Spectre de puissance}
\subsection{Fonction de corrélation angulaire}
\subsection{Spectre de puissance angulaire}

\section{Calcul théorique}
\subsection{Approximation de Limber}
\subsection{Exemples de noyaux}
QSO-kappa
QSO-Lyman
QSO-kappa

\section{Avantages de la corrélation croisée}
Elimination des bruits non corrélés
Amplification du signal
\subsection{Exemple théorique}
\subsection{Quelques résultats récents en cosmologie}