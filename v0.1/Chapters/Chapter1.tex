% Chapter Template

\chapter{Cosmologie moderne} % Main chapter title

\label{Chapter1} % Change X to a consecutive number; for referencing this chapter elsewhere, use \ref{ChapterX}

\lhead{Chapter 1. \emph{Modern Cosmology Overview}} % Change X to a consecutive number; this is for the header on each page - perhaps a shortened title

\epigraph{\textit{"Use the Force, Luke"}}{\par\raggedleft--- \textup{Master Yoda}, \textit{Star Wars}}



%----------------------------------------------------------------------------------------
%	SECTION 1
%----------------------------------------------------------------------------------------
\section{Historique}

\subsection{Relativité générale}
\subsection{Principe cosmologique}
\subsection{Métrique FLRW}
\subsection{Hubble et la récession des galaxies}
\subsection{L'abondance des éléments}
Alpher et Gamow 1948
\subsection{Le fond diffus cosmologique}
Harmoniques sphériques et SdPA
\subsection{Matière noire}
\subsection{Energie noire}


\section{Théorie du Big Bang et modèle $\Lambda$-CDM}
\subsection{Chronologie}
Histoire de l'univers depuis le Big Bang, les différentes phases :
Inflation \
Nucléosynthèse
Baryosynthèse
Ere de la radiation -> matière
Recombinaison
Ages sombres
Réionisation
Expansion
\subsection{Contenu énergétique de l'univers}
\subsubsection{Baryons}
\subsubsection{Photons}
\subsubsection{Matière noire}
\subsubsection{Energie noire}
\subsubsection{Neutrinos}

\subsection{Limites du modèle $\Lambda$-CDM}
\subsubsection{Problème de l'horizon}
\subsubsection{Problème de la courbure}
\subsubsection{Problème des monopôles}

\subsection{Paradigme de l'inflation}
\subsubsection{Postulats}
\subsubsection{Apports au modèle $\Lambda$-CDM}
Résolution des problèmes
Fluctuations qui deviennent des anisotropies

\subsubsection{Conséquences observationnelles}
