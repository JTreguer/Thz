% Chapter Template

\chapter{Corrélations avec sources X (BONUS !)} % Main chapter title

\label{Chapter8} % Change X to a consecutive number; for referencing this chapter elsewhere, use \ref{ChapterX}

\lhead{Chapter X. \emph{Chapter Title Here}} % Change X to a consecutive number; this is for the header on each page - perhaps a shortened title


\section{Données utilisées}
\subsection{Sources X XMM-Chandra}

\section{Méthodologies}

\subsection{Scatterplot}
\subsection{Rotations de ciel}
Fonction à un point
\subsection{Stacking}
\subsubsection{Description}
\subsubsection{Validation - Simulations}
\subsubsection{Validation - Null test}

\subsection{Spectres angulaires}
\subsubsection{Description}
\subsubsection{Validation - Simulations}
\subsubsection{Validation - Null test}


\section{Résultats expérimentaux}
\subsection{Stacking}
\subsection{Spectres angulaires}

\section{Estimation de paramètres}
\subsection{Fonction de corrélation angulaire théorique}
Courbe théorique du stacking
\subsection{Spectres angulaires théoriques}
\subsection{Résultats}
Quels paramètres ?
Faire un tableau comparant stacking et méthode harmonique

