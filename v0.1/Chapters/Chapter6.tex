% Chapter Template

\chapter{Corrélation avec catalogues de galaxies et de quasars} % Main chapter title

\label{Chapter6} % Change X to a consecutive number; for referencing this chapter elsewhere, use \ref{ChapterX}

\lhead{Chapter X. \emph{Chapter Title Here}} % Change X to a consecutive number; this is for the header on each page - perhaps a shortened title


\section{Données utilisées}

Dans cette section, nous précisons les données utilisées pour les analyses et les éventuels pré-traitements appliqués.

\subsection{Lensing du CMB}

Nous avons utilisé la carte du lensing de Planck version 2 (REF). Contrairement à la version 1 qui contenait le potentiel de lensing, la version 2 fournit directement  la convergence $\kappa$ sous forme d'une carte au format HEALPIX à haute résolution (nside = 2056).

La carte de lensing n'est pas une observation directe, elle est produite à partir de la carte de température du CMB de Planck. Il s'agit d'un estimateur du maximum de vraisemblance tenant compte des non-gaussianités induites par le champ de déflection sur le champ de température.


PRECISER FRACTION DE CIEL, masque utilisé et en 2 mots la méthodologie
AJOUTER FIGURE CARTE BRUTE + MASQUE et REFERENCE
COMPLETER AVEC LES MODIFICATIONS EVENTUELS

\subsection{NVSS}
Instrument, format, nombre d'échantillons, indicateurs de qualité

\subsection{Lowz et CMASS}
\subsection{Quasars de BOSS}

\section{Méthodologie}
\subsection{Stacking}

L

\subsection{Spectres angulaires}

\section{Validation}
\subsection{Simulation}
\subsection{Null test}

\section{Résultats expérimentaux}
\subsection{Lensing vs NVSS}
\subsection{Lensing vs Lowz et CMASS}
\subsection{Lensing vs Quasars de BOSS}

\section{Estimation de paramètres}
\subsection{Spectres angulaires théoriques}
\subsection{Résultats}
Biais par redshifts