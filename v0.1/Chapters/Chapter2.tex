% Chapter Template

\chapter{Lentillage gravitationnel du fonds diffus cosmologique} % Main chapter title

\label{Chapter2} % Change X to a consecutive number; for referencing this chapter elsewhere, use \ref{ChapterX}

\lhead{Chapter X. \emph{Chapter Title Here}} % Change X to a consecutive number; this is for the header on each page - perhaps a shortened title

\section{Fond diffus cosmologique}
\subsection{Spectre de puissance angulaire}

\section{Lentillage gravitationnel}
Définition, observations astrophysiques
\subsection{Origine physique}
\subsection{Théorie}
Cisaillement et magnification
\subsection{Applications}
Eddington et relativité générale
Détermination de masses
Détection de planètes

\section{Lentillage du CMB}
\subsection{Théorie}
Champ de déflection
Convergence $\kappa$

\subsection{Effets observationnels}
Effet sur la carte du CMB
Effet sur le spectre angulaire de température
Effet sur la polarisation du CMB